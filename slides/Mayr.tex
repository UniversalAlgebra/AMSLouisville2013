\documentclass{beamer}
\usepackage{amsmath,amssymb}
\usepackage{graphicx}
\usepackage{sidecap}
\usepackage{pgfarrows,pgfnodes}
\usepackage{url}
\usepackage{textcomp,alltt}

\usetheme{default}
\usecolortheme{dolphin}

\mode<presentation>
{   
   \setbeamercovered{transparent}
  \setcounter{secnumdepth}{-1}
}

\usepackage[english]{babel}

\setbeamertemplate{theorems}[normal font]


\newtheorem{thm}{Theorem}
\newtheorem{de}[thm]{Definition} 
\newtheorem{lem}[thm]{Lemma} 
\newtheorem{cor}[thm]{Corollary} 
\newtheorem{que}[thm]{Problem} 
\newtheorem{con}[thm]{Conjecture} 
\newtheorem{exa}[thm]{Example} 
\newtheorem{fa}[thm]{Fact} 
\newtheorem{rem}[thm]{Remark} 
\newtheorem{pro}[thm]{Proof} 
\theoremstyle{definition}

\newcommand{\Aut}{\mathrm{Aut}}
\newcommand{\End}{\mathrm{End}}
\newcommand{\id}{\mathrm{id}}
\renewcommand{\Im}{\mathrm{Im}}
\renewcommand{\span}{\mathrm{span}}


\newcommand{\algop}[2]{\langle {#1}, {#2} \rangle}
\newcommand{\ab}[1]{{\mathbf{#1}}}
\newcommand{\VComm}[3]{[\![ {#1}, {#2} ]\!]_{#3}}
\renewcommand{\vec}[2]{\left( \begin{array}{c} #1 \\ \vdots \\ #2 \end{array}\right)}
\newcommand{\vect}[2]{\left( \begin{array}{c} #1 \\ #2 \end{array}\right)}

\newcommand{\setsuchthat}{\ \pmb{|}\ } 
\newcommand{\Con}{\mathrm{Con}}
\newcommand{\Cg}{\mathrm{Cg}\,}
\newcommand{\AConC}{\mathbf{Con}^*}
\newcommand{\ACon}{\mathbf{Con}}
\newcommand{\cad}{\mathrm{cad}}
\newcommand{\CI}{(\mathrm{C1})}
\newcommand{\Comp}{\mathrm{Comp}}
\newcommand{\Clo}{\mathrm{Clo}}
\newcommand{\Pol}{\mathrm{Pol}}
\newcommand{\PSL}{\mathrm{PSL}}
\newcommand{\ord}{{\mathrm{ord}\, }}
\newcommand{\CSP}{\mathrm{CSP}}
\newcommand{\CP}{\mathrm{CP}}
\newcommand{\ch}{\mathrm{char}}
\newcommand{\F}{\mathcal{F}}
\newcommand{\Fr}{{\mathbf{F}}}
\newcommand{\AGL}{{\mathrm{AGL}}}
\newcommand{\Aff}{{\mathrm{Aff}}}
\newcommand{\Sub}{{\mathrm{Sub}}}
\newcommand{\SCI}{(\mathrm{SC1})}


\newcommand{\A}{{\mathbf A}}
\newcommand{\B}{{\mathbf B}}
\newcommand{\G}{{\mathbf G}}
\renewcommand{\H}{{\mathbf H}}
\newcommand{\R}{{\mathcal R}}
\renewcommand{\S}{{\mathbf S}}
\newcommand{\T}{{\mathbf T}}
\newcommand{\V}{{\mathcal V}}
\renewcommand{\P}{{\mathcal P}}
\newcommand{\Z}{{\mathbb Z}}
\newcommand{\N}{{\mathbb N}}

\newcommand{\mtt}[4]{\left(\begin{array}{cc} {#1} & {#2} \\ {#3} & {#4} \end{array}\right)}


\title[Direct products]{Finiteness properties of direct products}
\author{Peter Mayr \& Nik Ru\v{s}kuc}
\date{Louisville, October 5, 2013}
\institute{JKU Linz, Austria \\ University of St Andrews, UK }
%\titlegraphic{\includegraphics[width=3cm]{../JKULogolangengl}\hspace*{5cm}~\includegraphics[width=3cm]{../fwf-logo_var1}}


\begin{document}
\begin{frame}
\titlepage
\end{frame}



%%%%%%%%%%%%%%%%%%%%%%%%%%%%%%%%%%%%%%%%%%%%%%%%%%%%%%%%%%%%%%%%%%%%%%%%%
\section{Introduction}
%%%%%%%%%%%%%%%%%%%%%%%%%%%%%%%%%%%%%%%%%%%%%%%%%%%%%%%%%%%%%%%%%%%%%%%%%

\begin{frame}
 
\begin{block}{Nice Boring Theorem}
 $\A\times\B$ satisfies $\P$ iff $\A$ and $\B$ satisfy $\P$.
\end{block}

 Examples for property $\P$: \\
 being {\bf finitely generated, finitely presented, residually finite},...

%\pause
\bigskip

\begin{block}{Example}
\begin{enumerate}
\item
 {\bf Groups:} $\G\times\H$ is finitely generated iff $\G,\H$ are finitely generated
 (same for finitely presented, residually finite).
%\pause
\item
 {\bf Semigroups:} $\algop{\N}{+}$ is finitely generated but $\algop{\N}{+}^2$ is not.
\end{enumerate}
\end{block}

%\pause

\begin{que}
{\color{red} Which algebras and properties give Nice Boring Theorems?}
\end{que}
%\pause
 For semigroups see Robertson, Ru\v{s}kuc, Wiegold (1998), Gray, Ru\v{s}kuc, (2009).

\end{frame}


%%%%%%%%%%%%%%%%%%%%%%%%%%%%%%%%%%%%%%%%%%%%%%%%%%%%%%%%%%%%%%%%%%%%%%%%%
\section{Generation}
%%%%%%%%%%%%%%%%%%%%%%%%%%%%%%%%%%%%%%%%%%%%%%%%%%%%%%%%%%%%%%%%%%%%%%%%%

\begin{frame}
\frametitle{1. Finite generation}

\begin{lem}[Folklore]
 If $\A\times\B$ is finitely generated, then $\A$, $\B$ are finitely generated.
\end{lem}

\begin{proof}
 Being finitely generated is inherited by homomorphic images.
\end{proof}

%\pause

\begin{thm}
 Let $\A,\B$ in an idempotent variety [$t(x,\dots,x)\approx x$ for all terms $t$].

 Then $\A\times\B$ is finitely generated iff $\A,\B$ are finitely generated. 
\end{thm}

\begin{proof}
 If $X,Y$ generate $\A,\B$, then $X\times Y$ generates $\A\times\B$.
\end{proof}

%\pause

\begin{rem}
 We have a {\color{red}NBT for lattices} but not for their expansions:
 $\A := \algop{\N}{\max,\min,x+1}$ is generated by $1$, but $\A^2$ is not 
 finitely generated.  
\end{rem}
\end{frame}





\begin{frame}

\begin{thm}[Geddes, PhD-thesis]
 Let $\A,\B$ in a congruence permutable variety of finite signature $\F$.

 Then $\A\times\B$ is finitely generated iff $\A,\B$ are finitely generated.
\end{thm}


\begin{rem}
 Finite signature is necessary.

 $\A := \algop{\Z^\N}{+,-,\text{ all constants}}$ is generated by $\emptyset$
 but $\A^2$ is {\bf not} finitely generated.
\end{rem}
\end{frame}

\begin{frame}
\begin{block}{Proof, $\Leftarrow$.} 
 Let $X,Y$ generate $\A,\B$. Fix $u\in A, v\in B$. Define
\begin{eqnarray*}
  Z & := & X\times\{v\} \cup \{u\}\times Y \cup \{(u,v)\} \cup \\
    && \{(f^\A(u,\dots,u),v) \setsuchthat f\in\F \} \cup \\
    && \{(u, f^\B(v,\dots,v)) \setsuchthat f\in\F \}
\end{eqnarray*}
%\pause
{\color{blue} Claim:} $\forall a\in A\colon (a,v) \in \langle Z\rangle$
 
 Have term $s$ over $\F$ and $x_1,\dots,x_k\in X$: $s^\A(x_1,\dots,x_k) = a$.

 Induct on length of $s$:
\begin{enumerate}
\item If $s$ is a variable, then $a = x_i$ and $(a,v)\in Z$.
%\pause
\item Assume $s = f(t_1,\dots,t_n)$ for $f\in\F$, terms $t_1,\dots,t_n$.
 For $a_i := t_i^\A(x_1,\dots,x_k)$, we have $(a_i,v)\in\langle Z\rangle$. %\pause So
\[ \begin{array}{llll}
 (& f^\A(a_1,\dots,a_n), & f^\B(v,\dots,v)&) \in \langle Z\rangle \\
 (& u, & f^\B(v,\dots,v)&) \in Z \\
 (& u, & v &) \in Z
\end{array} \]
 Applying the Mal'cev term in each row yields $(a,v)\in\langle Z\rangle$.
\end{enumerate}
\end{block}
\end{frame}


\begin{frame}
\begin{block}{Proof, continued.}
 For all $a\in A, b\in B$
\[ \begin{array}{ll}
 (a,v) & \in \langle Z\rangle \\
 (u,v) & \in Z \\
 (u,b) & \in \langle Z \rangle
\end{array} \]
 Applying the Mal'cev term in each row yields $(a,b)\in\langle Z\rangle$.

 So $Z$ generates $\A\times\B$.\qed
\end{block}

\end{frame}




%%%%%%%%%%%%%%%%%%%%%%%%%%%%%%%%%%%%%%%%%%%%%%%%%%%%%%%%%%%%%%%%%%%%%%%%%
\section{Presentation}
%%%%%%%%%%%%%%%%%%%%%%%%%%%%%%%%%%%%%%%%%%%%%%%%%%%%%%%%%%%%%%%%%%%%%%%%%

 
\begin{frame}
\frametitle{2. Finite presentations}

\begin{de}
 $\A$ in a variety $\V$ is {\bf finitely presented} if
 $$\A\cong \Fr_\V(x_1,\dots,x_k) / \Cg((r_1,s_1),\dots,(r_n,s_n))$$
 for some $k,n\in\N$ and $(r_1,s_1),\dots,(r_n,s_n) \in\Fr_\V(x_1,\dots,x_k)^2$.
\end{de}

 In particular, free algebras over finite sets are finitely presented.

\bigskip

\begin{thm}
 Let $\V$ be the variety of loops with signature $(\cdot,\backslash,/,1)$.

 Then $\Fr_\V(x)\times\Fr_\V(x)$ is not finitely presented.
\end{thm}

%\pause

\begin{thm}
 Let $\V$ be the variety of lattices, ${\bf 2} := \algop{\{0,1\}}{\wedge,\vee}$.

 Then $\Fr_\V(x_1,x_2,x_3)\times{\bf 2}$ is not finitely presented.
\end{thm}
\end{frame}


\begin{frame}
\begin{block}{Proof, $\A\in\V$ is not finitely presented.}
\begin{enumerate}
\item
 Find $X$ finite and an onto homomorphism $h\colon\Fr_\V(X)\to\A$.
%\pause
\item
 Suppose $\ker h$ is generated by some $(r_1,s_1),\dots,(r_n,s_n)$.
%\pause
\item
 Find $u,v\in\Fr_\V(X)$ such that $h(u) = h(v)$ in $\A$ but
\[ u\not\equiv v \text{ in } \Fr_\V(X)/\Cg((r_1,s_1),\dots,(r_n,s_n)). \]
 Contradiction.
\end{enumerate}
%\pause
 For the {\color{red} word problem} in {\color{blue}3.} we use
\begin{itemize}
\item
 for loops: Evans' confluent rewriting systems (1951).
\item
 for lattices: Dean's solution of the word problem (1964). \qed
\end{itemize}
\end{block}
\end{frame}



%%%%%%%%%%%%%%%%%%%%%%%%%%%%%%%%%%%%%%%%%%%%%%%%%%%%%%%%%%%%%%%%%%%%%%%%%
\section{Residual Finiteness}
%%%%%%%%%%%%%%%%%%%%%%%%%%%%%%%%%%%%%%%%%%%%%%%%%%%%%%%%%%%%%%%%%%%%%%%%%


\begin{frame}
\frametitle{3. Residually finite}

\begin{de}
 $\A$ is {\bf residually finite} if for any distinct $a,b\in A$ there exist $\rho\in\Con(\A)$
 such that $A/\rho$ is finite and $a\not\equiv b \mod\rho$.
\end{de}
%\pause
\begin{lem}[Folklore]
 If $\A,\B$ are residually finite, then $\A\times\B$ is residually finite.
\end{lem}

\begin{proof}
 If $\alpha\in\Con(\A)$ separates $a_1,a_2$, then
 $\alpha \times 1_B \in \Con(\A\times\B)$ separates $(a_1,b_1), (a_2,b_2)$.
\end{proof}
 %\pause
\bigskip
 
 The converse holds for example
\begin{itemize}
\item if $\A,\B$ embed into  $\A\times\B$, \\
 {\color{red} NBT for algebras with idempotents (groups, monoids, lattices)}
%\pause
\item if $\Con(\A\times\B) = \Con(\A)\times\Con(\B)$. \\
 {\color{red} NBT for congruence distributive varieties}
\end{itemize}
\end{frame}




\begin{frame}

\begin{thm}
 Let $\A,\B$ in a congruence modular variety.

 Then $\A\times\B$ is residually finite iff $\A,\B$ are residually finite.
\end{thm}

%\pause

\begin{block}{Proof, $\Rightarrow$.}
 Let $a_1,a_2\in A$ be distinct, fix $b\in B$. \\
 Have $\rho\in\Con(\A\times\B)$ of finite index and $(a_1,b)\not\equiv_\rho (a_2,b)$.
%\pause
 Show
\[ \sigma := \{ (u,v)\in A^2 \setsuchthat \exists z\in B\colon (u,z) \equiv_\rho (v,z) \} \]
\vspace{-7mm}
\begin{itemize}
\item is a congruence on $\A$,
\item has finite index, and
\item separates $a_1,a_2$
\end{itemize}
 using commutators and a difference term.
\qed
\end{block}

\end{frame}



%%%%%%%%%%%%%%%%%%%%%%%%%%%%%%%%%%%%%%%%%%%%%%%%%%%%%%%%%%%%%%%%%%%%%%%%%
\section{Problems}
%%%%%%%%%%%%%%%%%%%%%%%%%%%%%%%%%%%%%%%%%%%%%%%%%%%%%%%%%%%%%%%%%%%%%%%%%


\begin{frame}
\frametitle{Problems} 

\begin{que}
 When is a subdirect product of finitely generated lattices finitely generated?
\end{que}

\begin{que}
 Characterize the finitely presented loops, lattices, \dots $\A,\B$ such that $\A\times\B$
 is finitely presented.
\end{que}

\begin{que}
 Is the following decidable:

 Given finitely presented semigroups
 $\A := \langle X\setsuchthat R\rangle, \B := \langle Y\setsuchthat S\rangle$. \\
 Is $\A\times\B$ finitely presented?
\end{que}

\begin{que}
 Does $\A\times\B$ residually finite imply that $\A,\B$ are residually finite
 in varieties with difference term?
\end{que}
\end{frame}
 



\section{AAA}


\begin{frame}
\frametitle{AAA87 -- Workshop on General Algebra}
 Johannes Kepler University Linz, Austria \\
 February 7--9, 2014

\setlength{\unitlength}{0.8cm}
\begin{picture}(12,7.5)
\put(0,6){Main speakers:}
\put(0,5){Andrei Bulatov}
\put(0,4.4){Agata Ciabattoni}
\put(0,3.8){Clemens Fuchs}
\put(0,3.2){Marcel Jackson}
\put(0,2.6){Ross Willard}
\put(0,2){Dmitriy Zhuk}
\put(4,0.5){\includegraphics[width=7.5cm]{linz1}}
\end{picture}

 For more information see http://www.jku.at/algebra/

\end{frame}
\end{document}
