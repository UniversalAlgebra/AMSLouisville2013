\documentclass[10pt, handout]{beamer}

%\usetheme{Goettingen}
\usetheme{Marburg}
\usecolortheme{whale}
\usefonttheme{professionalfonts}
\usepackage{amsmath, amssymb,amsthm,graphicx,float,ifthen,pb-diagram,pb-xy}
\usepackage[all]{xypic}
\usepackage{etex}



\renewcommand{\1}{1^\text{'}}
\newcommand{\0}{0^\text{'}}


\newcommand{\inv}{^{-1}}
\newcommand{\ds}{\displaystyle}
\newtheorem{thm}{Theorem}

\title{Finite Monk algebras and equational bases defining \textsf{RRA} over \textsf{wRRA}}
\author{Jeremy Alm}
\institute{Illinois College}
\date{October 6, 2013}


\begin{document}



\frame{
\titlepage
}
%%%%%%%%%%%%%%%%%%%%%%%%%%%%%%%%%%%%%%%%
\frame
{
This is joint work with Jacob Manske (Epic Systems Corporation) and Robin Hirsch (University College London).

}


%%%%%%%%%%%%%%%%%%%%%%%%%%%%%%%%%%%%%%%%%
\section{Representations of Monk Algebras}
\frame
{
A \emph{relation algebra} is an abstract algebra $\langle A,+,\cdot,^-,;,^\cup,1')$ satisfying several equational axioms.

\bigskip

An algebra $\underline{A}$ is \emph{representable} if there is an embedding $\underline{A}\rightarrow \langle P(E),\cup,\cap,^c,|,^{-1},Id\rangle$, where $E$ is some non-empty equivalence relation.  The class RRA of representable algebras is a non-finitely based variety.
}



%%%%%%%%%%%%%%%%%%%%%%%%%%%%%%%%%%%%%%%%%

\frame
{
A \emph{Monk algebra} (or Maddux algebra or Ramsey algebra) is a finite symmetric algebra with atoms $1',a_1,\ldots,a_n$ so that $$a_i;a_i=\overline{a_i}$$

and for $i\neq j$,
 $$a_i;a_j=0'=\overline{1'}.$$

\pause


\bigskip


equivalent characterization:
the only forbidden cycles of atoms (triangles) are the 1-cycles $a_ia_ia_i$ (monochromatic triangles)
}



%%%%%%%%%%%%%%%%%%%%%%%%%%%%%%%%%%%%%%%%%

\frame
{
\begin{center}


\includegraphics*[width=160pt]{Z13}



\end{center}
}



%%%%%%%%%%%%%%%%%%%%%%%%%%%%%%%%%%%%%%%%%

\frame
{
Is the Monk algebra $M(n)$ with $n$ diversity atoms representable?

\pause
\begin{itemize}
  \item Comer: YES for $n=2,3,4,5$ (late 80's)

  \pause
  \item Maddux: YES for $n=6,7$ ($\sim 2010$)

  \pause
  \item A., Manske: YES for $9\leq n\leq 300$ (except $n=13,292$)
\end{itemize}

\pause
(All cyclic group representations)
}

\frame
{Look at primes $N = nk + 1$ with $k$ even, find a generator $x$ of $\mathbb{Z}_{N}^{\times}$, and construct the partition
\[X_{0} = \left\{x^{0},x^{n},x^{2n},\ldots,x^{(k-1)n}\right\}\]
and $X_{i} = x\cdot X_{i-1},\ \text{for $i = 1,2,\ldots,n-1$}$.
\bigskip

\pause

Check that $X_{i}+X_{i}=\mathbb{Z}_{N}\setminus X_{i}$ and $X_{i} + X_{j} = \mathbb{Z}_{N}\setminus \left\{0\right\}$
}



%%%%%%%%%%%%%%%%%%%%%%%%%%%%%%%%%%%%%%%%%
\section{Weak representations of Monk algebras}
\frame
{A \emph{weak representation} is an embedding into $\langle P(E),\cup,\cap,^c,|,^{-1},Id\rangle$ that \emph{need not preserve} $\cup$ or $^c$.

\bigskip

\pause
Let wRRA denote the class of weakly representable algebras.
\begin{itemize}
  \item wRRA is a variety (Pesci 2009)
  \pause
  \item wRRA is not finitely based (Hodkinson-Mikulas 2000)
  \pause
  \item RRA is not finitely based over wRRA (Andreka 1994)
\end{itemize}

}

%%%%%%%%%%%%%%%%%%%%%%%%%%%%%%%%%%%%%

\frame
{
\begin{center}


\includegraphics*[width=70pt]{sigma}



\end{center}
}

%%%%%%%%%%%%%%%%%%%%%%%%%%%%%%%%%%%%%%%%%

\frame
{
\begin{thm}[A., Hirsch 2013]
   Let $\Sigma$ be an equational basis defining RRA over wRRA.  Then $\forall N\in\mathbb{Z}^+$ there is an equation $\varepsilon\in\Sigma$ with more than $N$ distinct variables.


\end{thm}

\pause
   Idea: Construct arbitrarily large weakly representable but not representable algebras whose ``small" subalgebras are all representable.
}



%%%%%%%%%%%%%%%%%%%%%%%%%%%%%%%%%%%%%%%%%

\frame
{
Proof sketch:
\begin{itemize}
  \item Suppose all equations in $\Sigma$ have at most $N$ distinct variables.
  \pause
  \item Consider $M(n),\, n>2^{N+2}$.
  \item Split one atom into $k=R_n(3)$ parts to ensure non-representability, get $M(n,k)$.
  \pause
  \item $M(n,k)$ is weakly representable (1-pt extension)
  \pause
  \item Any subalgebra $S$ generated by $N$ or fewer elements must have an atom $a$ that is above two or more ``unsplit" atoms.  Hence $a$ is flexible.
  \pause
  \item $S$ is representable over a countable set.
  \pause
  \item Hence $S \models\varepsilon$ for all such $S$.
  \item Therefore $M(n,k)\models \varepsilon$, and so $\Sigma$ cannot define RRA over wRRA.
\end{itemize}
}



%%%%%%%%%%%%%%%%%%%%%%%%%%%%%%%%%%%%%%%%%

\frame
{Future work:
Does wRRA have a finite-variable basis?  Presumably not.

\bigskip

Lyndon algebras from projective lines may decide the question.

}


\frame
{

Thanks to Roger Maddux for hosting Hirsch and me at Iowa State for several days in May.

}


















 

\end{document}