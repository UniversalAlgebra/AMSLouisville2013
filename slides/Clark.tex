\documentclass{beamer}
\usetheme{Berkeley}
%\usepackage{graphics}
\usepackage{amsmath,amsthm,latexsym,amssymb,verbatim}
\usepackage[mathscr]{eucal}
\usepackage{stmaryrd}
\usepackage{graphicx}
%%%%%%%%%%%%%%%%%%%%%%%%%%%%%%%%%%%%%%%%%%%%%%%%%%%%%%%%%%%%%%%%%%%%%%%%%%%%%%%%%%%%%%%%%%

\newcommand{\jn}{\vee}
\newcommand{\m}{\wedge}
\newcommand{\lz}{{\lozenge}}
\newcommand{\G}{{\rm{\textbf{G}}}}
\newcommand{\Bt}{{\rm{\textbf{B}}}}
\newcommand{\A}{{\mathbf A}}
\newcommand{\p}{\beta}
\newcommand{\bbF}{\mathbb F}
\newcommand{\bbM}{\mathbb M}
\newcommand{\N}{\mathbb N}
\newcommand{\T}{\mathscr T}
\newcommand{\B}{\beta}
\newcommand{\uc}{\uncover}

\newcommand{\va}{\vec a}
\newcommand{\vb}{\vec b}
\newcommand{\vd}{\vec{d}}
\newcommand{\vx}{\vec x}

\newcommand{\fit}[1]{\operatorname{fit}{#1}}
\newcommand{\Fit}[1]{\operatorname{Fit}{#1}}


%%%%%%%%%%%%%%%%%%%%%%%%%%%%%%%%%%%%%%%%%%%%%%%%%%%%%%%%%%%%%%%%%%%%%%%%%%%%%%%%%%%%%%%%%%

\newtheoremstyle{T}{11pt}{11pt}{\slshape}{}{\bfseries}{}{0.75em}{}
\newtheoremstyle{C}{11pt}{11pt}{\slshape}{}{\bfseries}{}{0.3em}{}
\newtheoremstyle{E}{11pt}{11pt}{\upshape}{}{\bfseries}{}{0.75em}{}
\newcommand{\citeskip}{\hspace*{0.75em}}







\begin{document}
\title {Algebraic Terms through Simulated \\Biological Evolution}
\author[D.Clark et al.]{David~M. Clark\inst{1} \and Maarten Keijzer\inst{2} \and Lee Spector\inst{3}}
\institute{
    \inst{1}{Mathematics:
             SUNY, New Paltz, NY, USA}\\
    \inst{2}{Machine Learning:
             Chordiant, Inc., Amsterdam, NDL}\\
    \inst{3}{Cognitive Science:
             Hampshire College, MA, USA}
             }
\date{October 6, 2013\\
${}$\\
 
$\langle$clarkd@newpaltz.edu$\rangle$
}

\begin{frame}
\titlepage
\end{frame}


\begin{comment}
\section[Outline]{Outline}

\begin{frame}
Outline:
\pause
\begin{enumerate}
   \item sample term generation problem\pause
   \item prior methods to solve term generation problems\pause
   \item evolutionary computation\pause
   \item main theorem:  conditions for EC to succeed\pause
   \item sample solution
\end{enumerate}
\end{frame}
\end{comment}



\section[Sample Problem]{}


\begin{frame}

A finite non-trivial groupoid $\G$ is \textbf{idemprimal} if  every operation $h:G^k \to G$ that preserves the idempotents of $\G$ is a term operation.
\pause   
\medskip

\centerline{$\Updownarrow$} 
\bigskip


 A finite non-trivial groupoid $\G$ is \textbf{idemprimal} if it has no non-trivial subalgebras, automorphisms or congruences and the ternary discriminator operation
\[
t(a,b,c) := 
\begin{cases}
   a  \mbox{ if } a=b\\
   c  \mbox{ if } a\neq b
\end{cases}
\]
is a term operation of $\G$.
\end{frame}







\begin{frame}

\textbf{Mursk\i {\u \i}'s Theorem.} (1975)  {\sl The proportion of $n$-element groupoids $\G = \langle \{0,1,2,\dots,n-1\},*\rangle$ that are idemprimal approaches 1 as $n \to \infty$.}
\pause
\bigskip


\textbf{Problem: } \\
Is this groupoid idemprimal?
\vspace{-10mm}

\centerline{
$\mathbf{Pi} := \langle\{0,1,2,3,4\}, *\rangle$ \qquad
    \begin{tabular}{c|ccccc}
    $*$ & 0 & 1 & 2 & 3 & 4 \\  \hline 
     0  & 1 & 4 & 1 & 0 & 4  \\
     1  & 2 & 1 & 0 & 3 & 0  \\
     2  & 3 & 4 & 2 & 4 & 3  \\
     3  & 2 & 3 & 3 & 4 & 1  \\
     4  & 2 & 1 & 4 & 3 & 3
    \end{tabular}
}
\pause

\vspace{-.9cm}
--  if and only if it has a \\
discriminator term
\pause
\bigskip

\textbf{Term Generation Problem: }  Given a finite groupoid $\G$ and an operation $h:G^k\to G$ that is a term operation of $\G$, find a term that represents~$h$.


%$\mathbf{Pi}$ is idemprimal if and only if it has a discriminator term.
\end{frame}









\section[Prior Methods]{}

\begin{frame}
Let $\G = \langle\{0,1,2,3,4\},*\rangle$ be a 5-element idemprimal groupoid.
Two prior methods might generate a discriminator term.
\pause
\begin{enumerate}
   \item UACalc will tell us if $\G$ is primal and, if so, will quickly produce a set of small terms from which we can recursively generate a discriminator term. \\  \pause
    BUT, the generated term will have $\approx 10^{10}$ variables(!)  \pause
   \item UACalc will construct the free algebra $\mathbf{F}_3(\G)$ and find a near minimal length term for the discriminator. \\   \pause
    BUT, if $\G$ has 2 idempotents, $\mathbf{F}_3(\G)$ will have $5^{123}$  elements. \pause 
Checking $10^6$ terms per second, the expected time to find a discriminator term will be $\approx 10^{72}$ years(!!)
 \end{enumerate}
\end{frame}





\section[Evolutionary Computation]{}

\begin{frame}
\textbf{Evolutionary computation} finds solutions to problems by simulating biological evolution. 
\medskip
\pause 

The \textbf{fitness} of a term $t(\vx)$ to represent $h:G^k \to G$ is 
\[
\fit(t(\vx)) := |\{\vb \in G^k :  t(\vb) = h(\vb)\}|.
\]
\pause
\vspace{-3mm}

\textbf{The Bucket Algorithm. }  Evolves two finite, dynamically changing sets of terms:
\pause
\medskip

$\mathbb M$, a set of \textbf{male terms}~$m(\vx)$;
\medskip
\pause

$\mathbb F$, a set of \textbf{female terms}~$f(\vx,\lz)$, each containing a single occurrence of the new variable $\lz$ (the \textbf{lozenge}).
\end{frame}






\begin{frame}
\begin{align*}
 &&&&  m(\vx)  && + && f(\vx,\lz)&& \implies &&& f(\vx,m(\vx))&&\ &&\\ 
 &&&& father    &&&&     mother &&                &&&  child &&  && &&
&
\end{align*}
\pause

The \textbf{fitness} of a male term $m(\vx)$ is 
\[
\Fit(m(\vx)) := Max\{\fit(f(\vx,m(\vx)) \mid f(\vx) \in \mathbb F\}.
\]
\pause

The \textbf{fitness} of a female term $f(\vx, \lz)$ is 
\[
\Fit(f(\vx,\lz)) := Max\{\fit(f(\vx,m(\vx)) \mid m(\vx) \in \mathbb M\}. 
\]

\end{frame}




\section{The Bucket Algorithm}

\begin{frame}
\centerline{\textbf{A Basic Bucket System}}
\pause

\begin{figure}[ht]
\setlength{\unitlength}{1mm}

\begin{picture}(120,48)
\put(-9,3){   %IJAC

% 2(n^k + 1) buckets
\multiput(10,0)(40,0){2}{
\put(20,28)
{\put(-2,4){\line(1,1){2}}  \put(0,0){\line(0,1){6}} \put(0,0){\line(1,0){20}} \put(20,0){\line(0,1){6}}  } 

\put(30,18){\circle*{.5}} \put(30,20){\circle*{.5}} \put(30,22){\circle*{.5}}

\put(20,10)
{\put(-2,4){\line(1,1){2}}  \put(0,0){\line(0,1){6}} \put(0,0){\line(1,0){20}} \put(20,0){\line(0,1){6}}  } 

\put(20,0)
{\put(-2,4){\line(1,1){2}}  \put(0,0){\line(0,1){6}} \put(0,0){\line(1,0){20}} \put(20,0){\line(0,1){6}}  } }


% box
\put(15,47){\line(1,0){90}}
\put(36,40){$\bbM$}   \put(81,40){$\bbF$}
\multiput(0,0)(0,40){2}{\put(15,-3){\line(1,0){90}}}  
\multiput(0,0)(45,0){3}{\put(15,-3){\line(0,1){50}}}  


% contents
\put(32,2){$x_0x_2$}  \put(44,2){$x_3$}  \put(34,12){$(x_4x_1)x_3^2$}  \put(39,30){?}
\put(75,2){$x_2(x_3\lz)$}  \put(80,12){$x_2\lz$}  \put(74,12){$\lz$}  \put(79,30){?}

% labels
\put(20,2){$\bbM_0$}   \put(20,12){$\bbM_1$}  \put(20,30){$\bbM_{n^k}$}
\put(95,2){$\bbF_0$}   \put(95,12){$\bbF_1$}  \put(95,30){$\bbF_{n^k}$}
 
}
\end{picture}
\end{figure}
\end{frame}
   
   
   
   
   
   
\begin{frame}
\centerline{\textbf{Initialize buckets.}}
\pause


\begin{figure}[ht]
\setlength{\unitlength}{1mm}

\begin{picture}(120,48)
\put(-9,3){   %IJAC

% 2(n^k + 1) buckets
\multiput(10,0)(40,0){2}{
\put(20,28)
{\put(-2,4){\line(1,1){2}}  \put(0,0){\line(0,1){6}} \put(0,0){\line(1,0){20}} \put(20,0){\line(0,1){6}}  } 

\put(30,18){\circle*{.5}} \put(30,20){\circle*{.5}} \put(30,22){\circle*{.5}}

\put(20,10)
{\put(-2,4){\line(1,1){2}}  \put(0,0){\line(0,1){6}} \put(0,0){\line(1,0){20}} \put(20,0){\line(0,1){6}}  } 

\put(20,0)
{\put(-2,4){\line(1,1){2}}  \put(0,0){\line(0,1){6}} \put(0,0){\line(1,0){20}} \put(20,0){\line(0,1){6}}  } }


% box
\put(15,47){\line(1,0){90}}
\put(36,40){$\bbM$}   \put(81,40){$\bbF$}
\multiput(0,0)(0,40){2}{\put(15,-3){\line(1,0){90}}}  
\multiput(0,0)(45,0){3}{\put(15,-3){\line(0,1){50}}}  


% contents
\put(32,2){$x_0x_2$}  \put(44,2){$x_3$}  
\put(78,2){$\lz$}


% labels
\put(20,2){$\bbM_0$}   \put(20,12){$\bbM_1$}  \put(20,30){$\bbM_{n^k}$}
\put(95,2){$\bbF_0$}   \put(95,12){$\bbF_1$}  \put(95,30){$\bbF_{n^k}$}
 
}
\end{picture}
\end{figure}

\end{frame}




\begin{frame}
\centerline{\textbf{Calculate fitness values.}}
\pause

\begin{figure}[ht]
\setlength{\unitlength}{1mm}

\begin{picture}(120,48)
\put(-9,3){   %IJAC

% 2(n^k + 1) buckets
\multiput(10,0)(40,0){2}{
\put(20,28)
{\put(-2,4){\line(1,1){2}}  \put(0,0){\line(0,1){6}} \put(0,0){\line(1,0){20}} \put(20,0){\line(0,1){6}}  } 

\put(30,18){\circle*{.5}} \put(30,20){\circle*{.5}} \put(30,22){\circle*{.5}}

\put(20,10)
{\put(-2,4){\line(1,1){2}}  \put(0,0){\line(0,1){6}} \put(0,0){\line(1,0){20}} \put(20,0){\line(0,1){6}}  } 

\put(20,0)
{\put(-2,4){\line(1,1){2}}  \put(0,0){\line(0,1){6}} \put(0,0){\line(1,0){20}} \put(20,0){\line(0,1){6}}  } }


% box
\put(15,47){\line(1,0){90}}
\put(36,40){$\bbM$}   \put(81,40){$\bbF$}
\multiput(0,0)(0,40){2}{\put(15,-3){\line(1,0){90}}}  
\multiput(0,0)(45,0){3}{\put(15,-3){\line(0,1){50}}}  


% contents
\put(37,12){$x_0x_2$}  \put(39,2){$x_3$}  
\put(79,12){$\lz$}

% labels
\put(20,2){$\bbM_0$}   \put(20,12){$\bbM_1$}  \put(20,30){$\bbM_{n^k}$}
\put(95,2){$\bbF_0$}   \put(95,12){$\bbF_1$}  \put(95,30){$\bbF_{n^k}$}
 
}
\end{picture}
\end{figure}

\end{frame}





\begin{frame}
\centerline{\textbf{Review configuration sometime later.}}
\pause

\begin{figure}[ht]
\setlength{\unitlength}{1mm}

\begin{picture}(120,48)
\put(-9,3){   %IJAC

% 2(n^k + 1) buckets
\multiput(10,0)(40,0){2}{
\put(20,28)
{\put(-2,4){\line(1,1){2}}  \put(0,0){\line(0,1){6}} \put(0,0){\line(1,0){20}} \put(20,0){\line(0,1){6}}  } 

\put(30,18){\circle*{.5}} \put(30,20){\circle*{.5}} \put(30,22){\circle*{.5}}

\put(20,10)
{\put(-2,4){\line(1,1){2}}  \put(0,0){\line(0,1){6}} \put(0,0){\line(1,0){20}} \put(20,0){\line(0,1){6}}  } 

\put(20,0)
{\put(-2,4){\line(1,1){2}}  \put(0,0){\line(0,1){6}} \put(0,0){\line(1,0){20}} \put(20,0){\line(0,1){6}}  } }


% box
\put(15,47){\line(1,0){90}}
\put(36,40){$\bbM$}   \put(81,40){$\bbF$}
\multiput(0,0)(0,40){2}{\put(15,-3){\line(1,0){90}}}  
\multiput(0,0)(45,0){3}{\put(15,-3){\line(0,1){50}}}  


% contents
\put(37,12){$x_0x_2$}  \put(34,2){$x_5, x_3, x_1$}  
\put(75,12){$x_2(\lz x_4)$}

% labels
\put(20,2){$\bbM_0$}   \put(20,12){$\bbM_1$}  \put(20,30){$\bbM_{n^k}$}
\put(95,2){$\bbF_0$}   \put(95,12){$\bbF_1$}  \put(95,30){$\bbF_{n^k}$}
 
}
\end{picture}
\end{figure}

\end{frame}





\section{Validity and Idemprimality}

\begin{frame}
%\centerline

Choose a new term, say $x_7$, and a female term, say $x_2(\lz x_4)$.  \pause
\medskip

Construct a new female term $x_2((x_7\lz)x_4)$ from $x_7$ and $x_2(\lz x_4)$.\pause
\medskip
 
We would add $x_2((x_7\lz)x_4)$ to $F_0$  with the goal of finding  $m(\vx)$ such that $x_2((x_7m(\vx))x_4)^\G = h$.
\pause
\medskip

Is there such a term $m(\vx)$?  \pause  If so, then $x_2((x_7\lz)x_4)$ must be \textbf{valid} with respect to $h$ in the sense that
\[
(\forall \va\in G^k) (\exists b\in G) :  a_2((a_7b)a_4) = h(\va), 
\]
namely, $b = m(\va)$.  
\medskip
\pause 

\textbf{Idemprimality Theorem. } {\it If\/ $\G$ is idemprimal, then there is a term $m(\vx)$ such that $f(\vx,m(\vx))^\G = h$ if and only if $f(\vx,\lz)$ is valid with respect to $h$.}   

\end{frame}







\begin{frame}
\centerline{\textbf{If $x_2((x_7\lz)x_4)$ is valid, add her to $\bbF_0$.}}
\pause

\begin{figure}[ht]
\setlength{\unitlength}{1mm}

\begin{picture}(120,48)
\put(-9,3){   %IJAC

% 2(n^k + 1) buckets
\multiput(10,0)(40,0){2}{
\put(20,28)
{\put(-2,4){\line(1,1){2}}  \put(0,0){\line(0,1){6}} \put(0,0){\line(1,0){20}} \put(20,0){\line(0,1){6}}  } 

\put(30,18){\circle*{.5}} \put(30,20){\circle*{.5}} \put(30,22){\circle*{.5}}

\put(20,10)
{\put(-2,4){\line(1,1){2}}  \put(0,0){\line(0,1){6}} \put(0,0){\line(1,0){20}} \put(20,0){\line(0,1){6}}  } 

\put(20,0)
{\put(-2,4){\line(1,1){2}}  \put(0,0){\line(0,1){6}} \put(0,0){\line(1,0){20}} \put(20,0){\line(0,1){6}}  } }


% box
\put(15,47){\line(1,0){90}}
\put(36,40){$\bbM$}   \put(81,40){$\bbF$}
\multiput(0,0)(0,40){2}{\put(15,-3){\line(1,0){90}}}  
\multiput(0,0)(45,0){3}{\put(15,-3){\line(0,1){50}}}  


% contents
\put(37,12){$x_0x_2$}  \put(34,2){$x_5, x_3, x_1$}  
%\put(34,12){$(x_4x_1)x_3^2$}  \put(39,30){?}
%\put(75,2){$x_2(x_3\lz)$}  \put(80,12){$x_2\lz$}  \put(74,12){$\lz$}  \put(79,30){?}
\put(75,12){$x_2(\lz x_4)$}  
\put(70.8,2){$x_2((x_1\lz)x_4)$}  %\put(66.5,2){?}  \put(91.5,2){?}
 
% labels
\put(20,2){$\bbM_0$}   \put(20,12){$\bbM_1$}  \put(20,30){$\bbM_{n^k}$}
\put(95,2){$\bbF_0$}   \put(95,12){$\bbF_1$}  \put(95,30){$\bbF_{n^k}$}
 
}
\end{picture}
\end{figure}

\end{frame}




\begin{comment}

\begin{frame}
\textbf{Then, recursively, start a new bucket system looking}

\textbf{for a solution to $x_2((x_7\lz)x_4)$.}
\pause

\begin{figure}[ht]
\setlength{\unitlength}{.5mm}

\begin{picture}(120,48)
\put(-25,-20){   %IJAC

% 2(n^k + 1) buckets
\multiput(10,0)(40,0){2}{
\put(20,28)
{\put(-2,4){\line(1,1){2}}  \put(0,0){\line(0,1){6}} \put(0,0){\line(1,0){20}} \put(20,0){\line(0,1){6}}  } 

\put(30,18){\circle*{.5}} \put(30,20){\circle*{.5}} \put(30,22){\circle*{.5}}

\put(20,10)
{\put(-2,4){\line(1,1){2}}  \put(0,0){\line(0,1){6}} \put(0,0){\line(1,0){20}} \put(20,0){\line(0,1){6}}  } 

\put(20,0)
{\put(-2,4){\line(1,1){2}}  \put(0,0){\line(0,1){6}} \put(0,0){\line(1,0){20}} \put(20,0){\line(0,1){6}}  } }

% box
\put(15,47){\line(1,0){90}}
\put(36,40){$\bbM$}   \put(81,40){$\bbF$}
\multiput(0,0)(0,40){2}{\put(15,-3){\line(1,0){90}}}  
\multiput(0,0)(45,0){3}{\put(15,-3){\line(0,1){50}}}  

% labels
\put(-2,0){
\put(20,2){$\bbM_0$}   \put(20,12){$\bbM_1$}  \put(18,30){$\bbM_{n^k}$}
\put(95,2){$\bbF_0$}   \put(95,12){$\bbF_1$}  \put(95,30){$\bbF_{n^k}$} 
}

\put(58,-15){$\lz$}
}



\put(90,20){   %IJAC
\setlength{\unitlength}{.2mm}
\tiny

% 2(n^k + 1) buckets
\multiput(17,0)(26,0){2}{
\put(20,28)
{\put(-2,4){\line(1,1){2}}  \put(0,0){\line(0,1){6}} \put(0,0){\line(1,0){20}} \put(20,0){\line(0,1){6}}  } 

\put(30,18){\circle*{.5}} \put(30,20){\circle*{.5}} \put(30,22){\circle*{.5}}

\put(20,10)
{\put(-2,4){\line(1,1){2}}  \put(0,0){\line(0,1){6}} \put(0,0){\line(1,0){20}} \put(20,0){\line(0,1){6}}  } 

\put(20,0)
{\put(-2,4){\line(1,1){2}}  \put(0,0){\line(0,1){6}} \put(0,0){\line(1,0){20}} \put(20,0){\line(0,1){6}}  } }

% box
\put(15,47){\line(1,0){90}}
\put(36,38){$\bbM$}   \put(81,38){$\bbF$}
\multiput(0,0)(0,40){2}{\put(15,-3){\line(1,0){90}}}  
\multiput(0,0)(45,0){3}{\put(15,-3){\line(0,1){50}}}  

% labels
\put(-2,0){
\put(20,2){$\bbM_0$}   \put(20,12){$\bbM_1$}  \put(18,27){$\bbM_{n^k}$}
\put(93,2){$\bbF_0$}   \put(92,12){$\bbF_1$}  \put(89,27){$\bbF_{n^k}$}
 }
 
 \normalsize
 
 \put(14,-27){$x_2((x_7\lz)x_4)$}
 }
\end{picture}
\end{figure}

\end{frame}

\end{comment}




\section{Continuity}

\begin{frame}


\begin{align*}
\lz \to x_0\lz &\to x_0(\lz x_3) \to x_0((\lz x_2^2)x_3)\\ 
                      &\to x_0((((x_4x_1)\lz)x_2^2)x_3) \\
                      &\to x_0((((x_4x_1)(\lz (x_5x_1^2)))x_2^2)x_3) \\
                      &\to x_0((((x_4x_1)((\lz(x_2^2x_0^2))(x_5x_1^2)))x_2^2)x_3) \to \dots
\end{align*}
\pause

 A finite groupoid $\G$ is \textbf{term continuous} if 
 \begin{gather*}
|\{f(\vx,m(\vx))^\G : m(\vx) \mbox{ a term}\}| \to 1\\
   \mbox{ as the depth of $\lz$ in } f(\vx,\lz) \to \infty.
 \end{gather*}
\pause

\textbf{Continuity Theorem. }  {\it A finite groupoid $\G$ is term continuous if it is asymptotically complete and has no subgroupoid with a separating relation.}  \pause  (IJAC, Vol. 23, No. 5 (2013) 1175-1205.)
\medskip

\end{frame}




\begin{frame}
\textbf{Conjecture 1. }  {\it Almost every finite groupoid is asymptotically complete.} 
\pause
\medskip

\textbf{Conjecture 2. }  {\it Almost every finite groupoid has no subgroupoid with a separating relation.}
\pause
\medskip

\textbf{Example. }  {\it Asked to find a discriminator term for the 5-element groupoid\/ $\mathbf{Pi}$, the Bucket Algorithm produced the following output.} \pause
\medskip

Length: \pause 793 variables ($ < 10^{10}$ variables)
\pause

Time: \pause 18 minutes ($ < 10^{72}$ years)
\end{frame}




\section{Example}

\begin{frame}
\tiny
t(x,y,z) = \pause
zxxyz*x**yy*xx**y*zx*yz**yx*y**xzx*y*z*x**yx*z**xxzx****zx*zx***z*z***y*zx*xyxy*****y**x***xx**zzzzx
\vspace{-1mm}
*z***xx**z**zzzxz*x*xxzzz**zxy***yz*y*xx*y*****xz***z*y*y**zx*xz****yx**z**zzzzz*y*zy***zz*xz****xx*
\vspace{-1mm}
*z**zzzxz*x*xxzzz**zxy***yz*yxz**zz*x***xx*y*****xz***z*y*y**zx*xz****xx**z**zzyy*z***z*y*y*zyx*z**x
\vspace{-1mm}
x*zz**y*x*x**yxx*y*xz***z*zzzx**x*z*yy**xxzx****x***zzzxxy*yx*xzz*xxx**y****y****xy*x*z*xz****xx**z*
\vspace{-1mm}
****yxxx*xx***yyz*zz****y***xz*x*xz*xy***x**zzyz**yz*x*x*y**xz**z***zyz*xxz***x***zx**zxx*xx*z*yy***
\vspace{-1mm}
**yz*y*x*y*yz***zzyx**yz*x*x*y**xz**z***xxyy****xx*z*zzzxz*xzyyy**z**z**y**yxzxxy****z***xz**z***xy*
\vspace{-1mm}
****yx*y*yz***zzzx**y****yy*y*z*zzxxx*xy***zx****xx*yx**x*x**yx**z***xx*****xx*xz**x*yxzy*zz*yx**yy
\vspace{-1mm}
*xz******xy*x**zzxxz*****yyx*yz**xy*z**z****x**xyxy***z*y*xzx*y*yx*z*xyxy***z*y***z*x*zxy**x*xz*xx**
\vspace{-1mm}
x**x****z*zy***yy*y*y*z*y**yz***yz*yxz*x**xz*x**y*z**z*y**zyxyy**xyzy***yz**xz*xyzy***yz***zyx**x*z
\vspace{-1mm}
yz**yx*zx*****x***z*y*y****yxyzzy*zz**z**xxzx*****xy*xx**x*x****xy***zyz*x*xxyx*xx*xzzx*xy**zz**y*y*
\vspace{-1mm}
y***z********zyx*y*z**yxxz*zxyy***z*yzz***yxz**zx*xz****zzz***zy*z*****x*zx*z****zzz*yx*y***zx*zx**z
\vspace{-1mm}
y*z**x*yy*zyyzyy*yy**x*******x*y**y**zzx*x**yyxz*xy***z*zyy*yy*x*x*y*xx**xxy***zz*xx**zy*x****x*xz*
\vspace{-1mm}
zx******x***zxzz**x**x*yy*y*xxxyx****zy**yzzy**x**zyxx*xy**yyz**y****x*zyx*y*z*xy*****xxxy***xz*z*y
\vspace{-1mm}
****yyzxy***zx*xz******xx*z*z*zz*y*yy*y*yz*x*y*y***y*zy*yzy******zxxzy***yyx***z*xy****yxz***z*z*zx*
\vspace{-1mm}
zz*****yx*x*yy*xy*xzxy*zy***zz*z*zz*y*z*y*z*zyxy**x*zz*z*x**zyzy*****yyyz***yx*x**y***zz*xx*yzy***zx
\vspace{-1mm}
*zz**xyx******yx*zzyx*zy**z******yxy*z******yxx******zxxxz*zzx**zx*z*zyzzy*y**xy*y*y***y*****z**z**
\vspace{-1mm}
x*x***zx*xzxzx*zxz****z***x*xx**zz***** %\pause \hspace{6cm} \small ${}$
% \hspace{2cm} 793 vbles ($< 10^{10}$ vbles) \quad \pause         18 min ($< %10^{72}$ yrs)  
\end{frame}






\section{Thanks!}

\begin{frame}

\vspace{1.5cm} \it

\hspace{2cm}\parbox{8cm}{\LARGE Thanks for staying to the\\ end of the conference! \bigskip

\hspace{4cm} --- DC}

\end{frame}



\section{References}

\begin{frame}
\textbf{References}
\begin{enumerate}

\item D. Clark, Evolution of algebraic terms 1:  term to term operation continuity, \textit{International Journal of Algebra and Computation}, Vol. 23, No. 5 (2013) 1175-1205.

\item D. Clark, B. Davey, J.Pitkethly, D.Rifqui, Flat unars:  the primal, the semi-primal and the dualisable , \textit{Algebra Universalis} Vol 63, No. 4 (2010), 303-329.

\item D. Clark, M. Keijzer and L. Spector, Evolution of algebraic terms 2:  evolutionary algorithms,  in preparation. 

   \item  L. Spector, D. Clark, B. Barr, J. Klein and I. Lindsay, Genetic programming for finite algebras, \emph{Genetic and Evolutionary Computation Conference} (GECCO) 2008 Proceedings, Atlanta GA (July 2008), (ACM), ISBN: 978-1-60558-130-9, 1291-1298.  [Won first place in 2008 ACM Hummie Competition.] 


\end{enumerate}
\end{frame}

%\end{document}















\begin{frame}
${}$
\end{frame}






\section[Separating Relations]{}

\begin{frame}
\textbf{Definition.}  A binary relation $\sigma$ on $G$ is a \textbf{separating relation} if, for all $a,b,c \in G$, 
\begin{enumerate}
   \item $\sigma \neq \varnothing$,
   \item $(a,b)\in \sigma$ implies $(b,a)\in \sigma$, 
   \item $(a,a) \notin \sigma$, 
   \item $(a,b) \in \sigma$ implies $(ac,bc)\in \sigma$ and $(ca,cb)\in \sigma$.
\end{enumerate} 
\medskip
\pause

For example,  if $\G$ is a non-trivial quasigroup, then $\neq$ is a separating relation on $\G$.
\medskip
\pause


\textbf{Lemma.} {\sl If a groupoid has a separating relation, then it is not term continuous.}

\end{frame}




\begin{frame}
\textbf{Test for Separating Relations.}
\medskip
\pause

$\mathbf{Pi} := \langle\{0,1,2,3,4\}, *\rangle$ \hspace{1.3cm}  Not in any separating relation.
\smallskip

\qquad
    \begin{tabular}{c|ccccc}
    $*$ & 0 & 1 & 2 & 3 & 4 \\  \hline 
     0  & 1 & 4 & 1 & 0 & 4  \\
     1  & 2 & 1 & 0 & 3 & 0  \\
     2  & 3 & 4 & 2 & 4 & 3  \\
     3  & 2 & 3 & 3 & 4 & 1  \\
     4  & 2 & 1 & 4 & 3 & 3
    \end{tabular}
    \hspace{1.0cm}
    \begin{tabular}{c|ccccc}
    $*$ & 0 & 1 & 2 & 3 & 4 \\  \hline 
     0  &   \uc<2->x &&&&  \\
     1  && \uc<2->x &&&  \\
     2  &&& \uc<2->x &&  \\
     3  &&&& \uc<2->x &\\
     4  &&&&& \uc<2->x
    \end{tabular}
\bigskip


\uc<2-> {${}$  \\

${}$}

\end{frame}






\begin{frame}
\textbf{Test for Separating Relations.}
\medskip

$\mathbf{Pi} := \langle\{0,1,2,3,4\}, *\rangle$ \hspace{1.3cm}  Not in any separating relation.
\smallskip

\qquad
    \begin{tabular}{c|ccccc}
    $*$ & 0 & 1 & 2 & \textcolor{red}{3} & \textcolor{red}{4} \\  \hline 
     0  & 1 & 4 & 1 & 0 & 4  \\
     1  & 2 & 1 & 0 & 3 & 0  \\
     2  & 3 & 4 & 2 & 4 & 3  \\
     3  & 2 & 3 & 3 & 4 & 1  \\
     \textcolor{red}{4}  & 2 & 1 & 4 & \textcolor{red}{3} & \textcolor{red}{3}
    \end{tabular}
    \hspace{1.0cm}
    \begin{tabular}{c|ccccc}
    $*$ & 0 & 1 & 2 & 3 & 4 \\  \hline 
     0  &  x &&&&  \\
     1  && x &&&  \\
     2  &&& x &&  \\
     3  &&&& x &\uc<2->{\textcolor{red}{x}}\\
     4  &&&&& x
    \end{tabular}
\bigskip


\uc<2-> {${}$  \\

${}$}

\end{frame}





\begin{frame}
\textbf{Test for Separating Relations.}
\medskip


$\mathbf{Pi} := \langle\{0,1,2,3,4\}, *\rangle$ \hspace{1.3cm}  Not in any separating relation.
\smallskip

\qquad
    \begin{tabular}{c|ccccc}
    $*$ & 0 & 1 & 2 & 3 & 4 \\  \hline 
     0  & 1 & 4 & 1 & 0 & 4  \\
     1  & 2 & 1 & 0 & 3 & 0  \\
     2  & 3 & 4 & 2 & 4 & 3  \\
     3  & 2 & 3 & 3 & 4 & 1  \\
     4  & 2 & 1 & 4 & 3 & 3
    \end{tabular}
    \hspace{1.0cm}
    \begin{tabular}{c|ccccc}
    $*$ & 0 & 1 & 2 & 3 & 4 \\  \hline 
     0  &  x & \uc<3->x & \uc<2->x & \uc<3->x & \uc<2->x  \\
     1  &   \uc<3->x & x & \uc<3->x & \uc<2->x & \uc<2->x  \\
     2  &   \uc<2->x & \uc<3->x & x & \uc<2->x & \uc<2->x  \\
     3  &   \uc<3->x & \uc<2->x & \uc<2->x & x & x  \\
     4  &   \uc<2->x & \uc<2->x & \uc<2->x & x & x
    \end{tabular}
\bigskip


\uc<4-> {Conclusion:  \\

$\mathbf{Pi}$ has no subalgebra with a separating relation.}

\end{frame}









\section[Asymptotic Completeness]{}

\begin{frame}
Let $\vd \in G^k$ and let $a\in G$.  For $H\in \N$ define 
\[
\beta_{\vd,a}(H) := \mbox{Prob}\langle t(\vd)=a \mid t \mbox{ is a term of height } \leq H\rangle.
\]
\bigskip
\pause

\textbf{Definition.}  The groupoid $\G$ is \textbf{asymptoticly complete} if, for each $k\in \N$, each $\vd \in \G^k$ and each $a \in G$, the sequence $\beta_{\vd,a}$ either has constant value $0$ or is eventually bounded away from~0.


\end{frame}




\begin{frame}
\textbf{Test for Asymptotic Completeness.}
\smallskip
\pause

$\mathbf{Pi}$ distributions with $k\in \N$ and  $\vd \in Pi^k$.
\vspace{-4mm}

\begin{align*}
 H               &&\B_{\vd,0}(H)&&\B_{\vd,1}(H)&&\B_{\vd,2}(H)&&\B_{\vd,3}(H)&&\B_{\vd,4}(H)   &&&&&\\
1	         &&			&&			&&			&&			&&		      &&&&&\\
2		&&			&&			&&			&&			&&		      &&&&&\\
4		&&			&&			&&			&&			&&		      &&&&&\\
8		&&			&&			&&			&&			&&		      &&&&&\\
6		&&			&&			&&			&&			&&		      &&&&&\\
32		&&			&&			&&			&&			&&		      &&&&&\\
64		&&			&&			&&			&&			&&		      &&&&&\\
500		&&			&&			&&			&&			&&		      &&&&&
\end{align*}

\end{frame}







\begin{frame}
\vspace{2.3mm}
\textbf{Test for Asymptotic Completeness.}
\vspace{1.2mm}

$\mathbf{Pi}$ distributions with $k=10$ and  $\vd = (2,2,2,2,2,2,2,2,2,2)$.
\vspace{-4mm}
\pause

\begin{align*}
 H               &&\B_{\vd,0}(H)&&\B_{\vd,1}(H)&&\B_{\vd,2}(H)&&\B_{\vd,3}(H)&&\B_{\vd,4}(H)   &&&&&\\
1	         &&	0		&&	0		&&	1		&&	0		&&	0	      &&&&&\\
2		&&	0		&&	0		&&	1		&&	0		&&	0	      &&&&&\\
4		&&	0		&&	0		&&	1		&&	0		&&	0	      &&&&&\\
6		&&	0		&&	0		&&	1		&&	0		&&	0	      &&&&&\\
8		&&	0		&&	0		&&	1		&&	0		&&	0	      &&&&&\\
10		&&	0		&&	0		&&	1		&&	0		&&	0	      &&&&&\\
16		&&	0		&&	0		&&	1		&&	0		&&	0	      &&&&&\\
500		&&	0		&&	0		&&	1		&&	0		&&	0	      &&&&&\\
\end{align*}

\end{frame}






\begin{frame}
\textbf{Test for Asymptotic Completeness.}
\smallskip

$\mathbf{Pi}$ distributions with $k=10$ and  $\vd = (3,4,3,0,2,3,3,0,1,3)$.
\vspace{-4mm}

\begin{align*}
 H               &&\B_{\vd,0}(H)&&\B_{\vd,1}(H)&&\B_{\vd,2}(H)&&\B_{\vd,3}(H)&&\B_{\vd,4}(H)   &&&&&\\
\uncover<2->{1&&	0.2		&&	0.1		&&	0.1		&&	0.5		&&	0.1	      &&&&&\\}
\uncover<3->{2	&& 0.127273	&& 0.127273	&& 0.145455	&& 0.254545	&& 0.345455 &&&&&\\}
\uncover<4->{4	&& 0.098817	&& 0.182526	&& 0.088076	&& 0.368477	&& 0.262102 &&&&&\\}
\uncover<5->{8	&& 0.103371	&& 0.195561	&& 0.091358	&& 0.365539	&& 0.244171 &&&&&\\}
\uncover<6->{16&& 0.103372	&& 0.195275	&& 0.091609	&& 0.365434	&& 0.244311 &&&&&\\}\uncover<7->{32&& 0.103372	&& 0.195275	&& 0.091609	&& 0.365434	&& 0.244310 &&&&&\\}
\uncover<8->{64&& 0.103372	&& 0.195275	&& 0.091609	&& 0.365434	&& 0.244310 &&&&&\\}
\uncover<9->{500	&& 0.103372	&& 0.195275	&& 0.091609	&& 0.365434	&& 0.244310 &&&&&}
\end{align*}
\end{frame}



\begin{frame}
\vspace{2.3mm}
\textbf{Test for Asymptotic Completeness.}
\vspace{1.2mm}

$\mathbf{Pi}$ distributions with $k=10$ and  $\vd = (2,2,2,2,2,2,2,2,2,2)$.
\vspace{-4mm}

\begin{align*}
 H               &&\B_{\vd,0}(H)&&\B_{\vd,1}(H)&&\B_{\vd,2}(H)&&\B_{\vd,3}(H)&&\B_{\vd,4}(H)   &&&&&\\
1	         &&	0		&&	0		&&	1		&&	0		&&	0	      &&&&&\\
2		&&	0		&&	0		&&	1		&&	0		&&	0	      &&&&&\\
4		&&	0		&&	0		&&	1		&&	0		&&	0	      &&&&&\\
6		&&	0		&&	0		&&	1		&&	0		&&	0	      &&&&&\\
8		&&	0		&&	0		&&	1		&&	0		&&	0	      &&&&&\\
10		&&	0		&&	0		&&	1		&&	0		&&	0	      &&&&&\\
16		&&	0		&&	0		&&	1		&&	0		&&	0	      &&&&&\\
500		&&	0		&&	0		&&	1		&&	0		&&	0	      &&&&&\\
\end{align*}

\end{frame}







\begin{frame}
\vspace{2.3mm}
\textbf{Test for Asymptotic Completeness.}
\vspace{1.2mm}

$\mathbf{Pi}$ distributions with $k=10$ and  $\vd = (2,2,2,2,2,2,4,2,2,2)$.
\vspace{-4mm}

\begin{align*}
 H               &&\B_{\vd,0}(H)&&\B_{\vd,1}(H)&&\B_{\vd,2}(H)&&\B_{\vd,3}(H)&&\B_{\vd,4}(H)   &&&&&\\
\uncover<2->{1	   && 0.0		&&	0.0		&& 0.9		&&	0.0		&&	0.1	      &&&&&\\}
\uncover<3->{2   && 0.000000	&& 0.000000	&& 0.810001	&&    0.099999  &&	0.090000&&&&&\\}	
\uncover<4->{4   && 0.007380	&& 0.029583	&& 0.430369	&&    0.283600  &&	0.248968&&&&&\\}
\uncover<5->{8   && 0.095683	&& 0.211801	&& 0.070920	&&    0.382629	&&    0.238968&&&&&\\}
\uncover<6->{16 && 0.103370	&& 0.195268	&& 0.091610	&&    0.365436	&&    0.244316&&&&&\\}
\uncover<7->{32 && 0.103372	&& 0.195275	&& 0.091609	&&    0.365434	&&    0.244310&&&&&\\}
\uncover<8->{64 && 0.103372	&& 0.195275	&& 0.091609	&&    0.365434	&&    0.244310&&&&&\\}
\uncover<9->{500&&0.103372	&& 0.195275	&& 0.091609	&&    0.365434	&&    0.244310&&&&&}
\end{align*}

\uncover<10->{Experimental evidence that $\mathbf{Pi}$ is asymptoticly complete.}

\end{frame}


\end{document}





