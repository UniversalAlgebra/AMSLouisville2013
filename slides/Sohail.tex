
\documentclass[notes=show]{beamer}
%%%%%%%%%%%%%%%%%%%%%%%%%%%%%%%%%%%%%%%%%%%%%%%%%%%%%%%%%%%%%%%%%%%%%%%%%%%%%%%%%%%%%%%%%%%%%%%%%%%%%%%%%%%%%%%%%%%%%%%%%%%%%%%%%%%%%%%%%%%%%%%%%%%%%%%%%%%%%%%%%%%%%%%%%%%%%%%%%%%%%%%%%%%%%%%%%%%%%%%%%%%%%%%%%%%%%%%%%%%%%%%%%%%%%%%%%%%%%%%%%%%%%%%%%%%%
\usepackage{amssymb}
\usepackage{amsmath}
\usepackage{mathpazo}
\usepackage{hyperref}
\usepackage{multimedia}

\setcounter{MaxMatrixCols}{10}
%TCIDATA{OutputFilter=LATEX.DLL}
%TCIDATA{Version=5.50.0.2953}
%TCIDATA{<META NAME="SaveForMode" CONTENT="1">}
%TCIDATA{BibliographyScheme=Manual}
%TCIDATA{Created=Wednesday, February 02, 2011 22:34:19}
%TCIDATA{LastRevised=Tuesday, October 15, 2013 19:43:14}
%TCIDATA{<META NAME="GraphicsSave" CONTENT="32">}
%TCIDATA{<META NAME="DocumentShell" CONTENT="Other Documents\SW\Slides - Beamer">}
%TCIDATA{Language=American English}
%TCIDATA{CSTFile=beamer.cst}

\newenvironment{stepenumerate}{\begin{enumerate}[<+->]}{\end{enumerate}}
\newenvironment{stepitemize}{\begin{itemize}[<+->]}{\end{itemize} }
\newenvironment{stepenumeratewithalert}{\begin{enumerate}[<+-| alert@+>]}{\end{enumerate}}
\newenvironment{stepitemizewithalert}{\begin{itemize}[<+-| alert@+>]}{\end{itemize} }
\usetheme{Madrid}
\input{tcilatex}
\begin{document}

\title[Epis of posemigroup]{{\Large Epimorphisms in certain varieties of
partially ordered semigroups}}
\subtitle{{\small AMS fall southeastern sectional meeting 5--6 October 2013,
Louisville, USA}}
\author{Sohail Nasir}
\institute[Tartu - Waterloo]{Institute of Mathematics, University of Tartu, Estonia\\
Department of Pure Mathematics, University of Waterloo, Canada}
\date{06 October 2013}
\maketitle

%TCIMACRO{\TeXButton{BeginFrame}{\begin{frame}}}%
%BeginExpansion
\begin{frame}%
%EndExpansion

\QTR{frametitle}{Motivation}

\begin{stepitemize}
\item Let $U$ be a subsemigroup of a semigroup $S$. Then $d\in S$ is said to
be in the \textit{dominion} of $U$ if for every pair of semigroup
homomorphisms $f,g:S\longrightarrow T$ with $f\left\vert _{U}\right.
=g\left\vert _{U}\right. $ we have $f(d)=g(d)$.\bigskip

\item The set of all such elements of $S$ is called the dominion of $U$ is $%
S $, denoted by $dom_{S}(U)$. This is a subsemigroup of $S$.\bigskip

\item Clearly $U\subseteq dom_{S}(U)$.\bigskip

\item $U$ is said to be \textit{closed} in $S$ if $dom_{S}(U)=U$. We term $U$
\textit{absolutely closed} if it is closed in all of its semigroup
extensions.
\end{stepitemize}

%TCIMACRO{\TeXButton{Transition: Box Out}{\transboxout}}%
%BeginExpansion
\transboxout%
%EndExpansion
%TCIMACRO{\TeXButton{EndFrame}{\end{frame}}}%
%BeginExpansion
\end{frame}%
%EndExpansion

%TCIMACRO{\TeXButton{BeginFrame}{\begin{frame}}}%
%BeginExpansion
\begin{frame}%
%EndExpansion

\QTR{frametitle}{Motivation}

\begin{stepitemize}
\item Dominions are related with epimorphisms; in that $f:S\longrightarrow T$
is an epimorphism iff $dom_{T}(\func{Im}f)=T$.\bigskip

\item Consequently, a variety of semigroups is absolutely closed iff its
epimorphisms are surjective.
\end{stepitemize}

\begin{equation*}
\end{equation*}

%TCIMACRO{\TeXButton{Transition: Box Out}{\transboxout}}%
%BeginExpansion
\transboxout%
%EndExpansion
%TCIMACRO{\TeXButton{EndFrame}{\end{frame}}}%
%BeginExpansion
\end{frame}%
%EndExpansion

%TCIMACRO{\TeXButton{BeginFrame}{\begin{frame}}}%
%BeginExpansion
\begin{frame}%
%EndExpansion

\QTR{frametitle}{Motivation}

In 1966 J.R. Isbell proved the following theorem.

\begin{theorem}
$d\in dom_{S}(U)$ iff there exists a system of equalities (called a zigzag)%
\begin{equation*}
\begin{tabular}{lll}
$d=s_{1}u_{1}$ &  & $u_{1}=v_{1}t_{1}$ \\ 
$s_{1}v_{1}=s_{2}u_{2}$ &  & $u_{2}t_{1}=v_{2}t_{2}$ \\ 
$\vdots $ &  & $\vdots $ \\ 
$s_{n-1}v_{n-1}=u_{n}$ &  & $u_{n}t_{n-1}=d$%
\end{tabular}%
\end{equation*}%
with $s_{1},\ldots ,s_{n-1};t_{1},\ldots ,t_{n-1}\in S$, $u_{1},\ldots
,u_{n};v_{1},\ldots ,v_{n-1}\in U$.
\end{theorem}

\begin{equation*}
\end{equation*}

%TCIMACRO{\TeXButton{Transition: Box Out}{\transboxout}}%
%BeginExpansion
\transboxout%
%EndExpansion
%TCIMACRO{\TeXButton{EndFrame}{\end{frame}}}%
%BeginExpansion
\end{frame}%
%EndExpansion

%TCIMACRO{\TeXButton{BeginFrame}{\begin{frame}}}%
%BeginExpansion
\begin{frame}%
%EndExpansion

\QTR{frametitle}{Motivation}

The aim of this talk is\bigskip

\begin{stepenumerate}
\item to present an analogous zigzag theorem for partially ordered
semigroups, briefly posemigroups and\bigskip

\item discuss closure properties and epimorphisms in certain varieties of
posemigroups.
\end{stepenumerate}

\begin{equation*}
\end{equation*}

%TCIMACRO{\TeXButton{Transition: Box Out}{\transboxout}}%
%BeginExpansion
\transboxout%
%EndExpansion
%TCIMACRO{\TeXButton{EndFrame}{\end{frame}}}%
%BeginExpansion
\end{frame}%
%EndExpansion

%TCIMACRO{\TeXButton{BeginFrame}{\begin{frame}}}%
%BeginExpansion
\begin{frame}%
%EndExpansion

\QTR{frametitle}{Definitions}

\begin{stepitemize}
\item A partially ordered semigroup, briefly \textit{posemigroup}, is a pair 
$(S,\leq )$ comprising a semigroup $S$ and a partial order $\leq $ (on $S$)
that is \textit{compatible} with the binary operation, i.e. for all $%
s_{1},s_{2},t_{1},t_{2}\in S$,%
\begin{equation*}
(s_{1}\leq t_{1},s_{2}\leq t_{2})\Longrightarrow s_{1}s_{2}\leq t_{1}t_{2}.
\end{equation*}%
\smallskip

\item A posemigroup \textit{homomorphism} $f:(S,\leq _{S})\longrightarrow
(T,\leq _{T})$ is a monotone semigroup homomorphism.
\end{stepitemize}

%TCIMACRO{\TeXButton{Transition: Box Out}{\transboxout}}%
%BeginExpansion
\transboxout%
%EndExpansion
%TCIMACRO{\TeXButton{EndFrame}{\end{frame}}}%
%BeginExpansion
\end{frame}%
%EndExpansion

%TCIMACRO{\TeXButton{BeginFrame}{\begin{frame}}}%
%BeginExpansion
\begin{frame}%
%EndExpansion

\QTR{frametitle}{Definitions}

\begin{stepitemize}
\item Let $(U,\leq _{U})$ be a subposemigroup of a posemigroup $(S,\leq
_{S}) $. Then the subposemigroup%
\begin{equation*}
\{x\in S:\alpha ,\beta :(S,{\small \leq }_{S})\rightarrow (T,{\small \leq }%
_{T})\text{, }\alpha \left\vert _{U}\right. =\beta \left\vert _{U}\right.
\Longrightarrow \alpha (x)=\beta (x)\}\text{ \ \ \ \ \ }
\end{equation*}%
is called the \textit{dominion} of $(U,\leq _{U})$ in $(S,\leq _{S})$, where 
$\alpha $ and $\beta $ are posemigroup homomorphisms.\bigskip

\item We shall denote this set by $\widehat{dom}_{S}(U)$.
\end{stepitemize}

%TCIMACRO{\TeXButton{Transition: Box Out}{\transboxout}}%
%BeginExpansion
\transboxout%
%EndExpansion
%TCIMACRO{\TeXButton{EndFrame}{\end{frame}}}%
%BeginExpansion
\end{frame}%
%EndExpansion

%TCIMACRO{\TeXButton{BeginFrame}{\begin{frame}}}%
%BeginExpansion
\begin{frame}%
%EndExpansion

\QTR{frametitle}{Definitions}

\begin{stepitemize}
\item A subposemigroup $(U,\leq _{U})$ of a posemigroup $(S,\leq _{S})$ will
be termed closed if $\widehat{dom}_{S}(U)=U$.\bigskip

\item $(U,\leq _{U})$ will be called absolutely closed if it is closed in
all of its posemigroup extensions.
\end{stepitemize}

\begin{equation*}
\end{equation*}

%TCIMACRO{\TeXButton{Transition: Box Out}{\transboxout}}%
%BeginExpansion
\transboxout%
%EndExpansion
%TCIMACRO{\TeXButton{EndFrame}{\end{frame}}}%
%BeginExpansion
\end{frame}%
%EndExpansion

%TCIMACRO{\TeXButton{BeginFrame}{\begin{frame}}}%
%BeginExpansion
\begin{frame}%
%EndExpansion

\QTR{frametitle}{Zigzag theorem for posemigroups}

\begin{theorem}
\label{Zigzag theorem for posemigroups}$d\in \widehat{dom}_{S}(U)$ if and
only if there exists a system of inequalities%
\begin{equation*}
\begin{tabular}{lll}
$d\leq s_{1}u_{1}$ &  & $u_{1}\leq v_{1}t_{1}$ \\ 
$s_{1}v_{1}\leq s_{2}u_{2}$ &  & $u_{2}t_{1}\leq v_{2}t_{2}$ \\ 
$\vdots $ &  & $\vdots $ \\ 
$s_{n-1}v_{n-1}\leq u_{n}$ &  & $u_{n}t_{n-1}\leq d$ \\ 
$v_{n}\leq s_{n+1}u_{n+1}$ &  & $d\leq v_{n}t_{n+1}$ \\ 
$s_{n+1}v_{n+1}\leq s_{n+2}u_{n+2}$ &  & $u_{n+1}t_{n+1}\leq v_{n+1}t_{n+2}$
\\ 
$\vdots $ &  & $\vdots $ \\ 
$s_{n+m}v_{n+m}\leq d$ &  & $u_{n+m}t_{n+m}\leq v_{n+m}$%
\end{tabular}%
\end{equation*}%
where $s_{i},t_{i}\in S$ and $u_{i},v_{i}\in U$.
\end{theorem}

%TCIMACRO{\TeXButton{Transition: Box Out}{\transboxout}}%
%BeginExpansion
\transboxout%
%EndExpansion
%TCIMACRO{\TeXButton{EndFrame}{\end{frame}}}%
%BeginExpansion
\end{frame}%
%EndExpansion

%TCIMACRO{\TeXButton{BeginFrame}{\begin{frame}}}%
%BeginExpansion
\begin{frame}%
%EndExpansion

\QTR{frametitle}{Remarks}

\begin{stepitemize}
\item We can also treat a posemigroup $(S,\leq )$ as a semigroup by
disregarding the order. We shall then denote it simply by $S$.\bigskip

\item So, given a subposemigroup $(U,\leq _{U})$ of a posemigroup $(S,\leq
_{S})$, one may also consider the (algebraic) dominion $dom_{S}(U)$%
{\normalsize \ }of $U$ in $S$.\bigskip

\item Because every equality in Isbell's zigzag can be treated as an
inequality, one can easily write from a `zigzag of equalities' a `zigzag of
inequalities'. We therefore have%
\begin{equation*}
d\in dom_{S}(U)\Longrightarrow d\in \widehat{dom}_{S}(U).
\end{equation*}%
\smallskip 

\item More precisely: $\ \ U\subseteq dom_{S}(U)\subseteq \widehat{dom}%
_{S}(U)\subseteq S$.\bigskip 
\end{stepitemize}

\begin{stepitemizewithalert}
\item So the closure of $(U,\leq _{U})$ in $(S,\leq _{S})$ implies the
closure of $U$ in $S$.
\end{stepitemizewithalert}

%TCIMACRO{\TeXButton{Transition: Box Out}{\transboxout}}%
%BeginExpansion
\transboxout%
%EndExpansion
%TCIMACRO{\TeXButton{EndFrame}{\end{frame}}}%
%BeginExpansion
\end{frame}%
%EndExpansion

%TCIMACRO{\TeXButton{BeginFrame}{\begin{frame}}}%
%BeginExpansion
\begin{frame}%
%EndExpansion

\QTR{frametitle}{The closure properties}

The following result tells that the converse of the above statement is also
true.\bigskip

\begin{stepitemize}
\item \textbf{Proposition} A subposemigroup $(U,\leq _{U})$ is closed in a
posemigroup $(S,\leq _{S})$ if and only if $U$ is such in $S$ as a
semigroup.\bigskip
\end{stepitemize}

\begin{stepitemizewithalert}
\item \textbf{Corollary} A posemigroup $(U,\leq )$ is absolutely closed if
and only if it is such as a semigroup within the class of semigroups that
qualify as posemigroup extensions of $(U,\leq )$, for certain compatible
partial orders.
\end{stepitemizewithalert}

%TCIMACRO{\TeXButton{Transition: Box Out}{\transboxout}}%
%BeginExpansion
\transboxout%
%EndExpansion
%TCIMACRO{\TeXButton{EndFrame}{\end{frame}}}%
%BeginExpansion
\end{frame}%
%EndExpansion

%TCIMACRO{\TeXButton{BeginFrame}{\begin{frame}}}%
%BeginExpansion
\begin{frame}%
%EndExpansion

\QTR{frametitle}{Absolutely closed varieties (\textbf{unordered context})}

Recall that a semigroup $S$ is called a right group if every equation $ax=b$
with $a,b\in S$ has a unique solution in $S$.\bigskip

Left groups are defined similarly\bigskip

\begin{theorem}[Higgins]
The absolutely closed varieties of semigroups are exactly the varieties
consisting entirely of semilattices of groups, or entirely of right groups
or entirely of left groups.
\end{theorem}

%TCIMACRO{\TeXButton{Transition: Box Out}{\transboxout}}%
%BeginExpansion
\transboxout%
%EndExpansion
%TCIMACRO{\TeXButton{EndFrame}{\end{frame}}}%
%BeginExpansion
\end{frame}%
%EndExpansion

%TCIMACRO{\TeXButton{BeginFrame}{\begin{frame}}}%
%BeginExpansion
\begin{frame}%
%EndExpansion

\QTR{frametitle}{Definition}

\begin{stepitemize}
\item A class of posemigroups is called a \textit{variety }(of posemigroups)
if it is closed under taking products (endowed with componentwise order),
homomorphic images (where of course we are only considering the monotone
semigroup homomorphisms) and subposemigroups.\bigskip

\item It is also possible to describe posemigroup varieties alternatively
with the help of inequalities using a Birkhoff type
characterization.\bigskip 

\item \textbf{Example} The class of all bounded posemigroups is a variety of
posemigroups defined by $\{(xy)z=x(yz)\}\cup \{x\leq a$ ($b\leq x$)$\}$.
Also, if $\mathcal{V}$ is a class of posemigroups that forms a variety of
semigroups (if the orders are disregarded), then the subclass $\mathcal{V}%
^{\prime }$ of all bounded posemigroups in $\mathcal{V}$ is a variety of
posemigroups defined by the identities of $\mathcal{V}$ and the
inequality(s) $x\leq a$ ($b\leq x$), provided $\mathcal{V}^{\prime }$ is
non-empty.
\end{stepitemize}

%TCIMACRO{\TeXButton{Transition: Box Out}{\transboxout}}%
%BeginExpansion
\transboxout%
%EndExpansion
%TCIMACRO{\TeXButton{EndFrame}{\end{frame}}}%
%BeginExpansion
\end{frame}%
%EndExpansion

%TCIMACRO{\TeXButton{BeginFrame}{\begin{frame}}}%
%BeginExpansion
\begin{frame}%
%EndExpansion

\QTR{frametitle}{Absolutely closed varieties (\textbf{ordered context})}

From the above corollary, viz,

\begin{stepitemizewithalert}
\item \textbf{Corollary} A posemigroup $(U,\leq )$ is absolutely closed if
and only if it is such as a semigroup within the class of semigroups that
qualify as posemigroup extensions of $(U,\leq )$, for certain compatible
partial orders,
\end{stepitemizewithalert}

\qquad and Higgins' theorem we obtian the following result.\bigskip

\begin{stepitemize}
\item \textbf{Corollary} Any variety $\mathcal{V}$ of posemigroups obtained
by endowing a subclass of a class of semigroups of Higgins' theorem with
compatible orders will be absolutely closed (in order theoretic sense).
\end{stepitemize}

%TCIMACRO{\TeXButton{Transition: Box Out}{\transboxout}}%
%BeginExpansion
\transboxout%
%EndExpansion
%TCIMACRO{\TeXButton{EndFrame}{\end{frame}}}%
%BeginExpansion
\end{frame}%
%EndExpansion

%TCIMACRO{\TeXButton{BeginFrame}{\begin{frame}}}%
%BeginExpansion
\begin{frame}%
%EndExpansion

\QTR{frametitle}{Absolutely closed varieties (\textbf{ordered context})}

\begin{stepitemize}
\item \textbf{Example} The class of all semilattices with natural orders
forms an absolutely closed variety of posemigroups. The assertion also
remains true if we substitute all the orders by their duals.\bigskip 

\item \textbf{Example} The classes of bounded posemigroups that are right
groups or left groups or semilattices of groups are absolutely closed (in
ordered context).\bigskip 

\item \textbf{Question }Are there any (order theoretic) varieties of
absolutely closed posemigroups other then those of the above corollary?
Especially are there any order theoretic varieties `bigger' than those of
Higgins'.
\end{stepitemize}

%TCIMACRO{\TeXButton{Transition: Box Out}{\transboxout}}%
%BeginExpansion
\transboxout%
%EndExpansion
%TCIMACRO{\TeXButton{EndFrame}{\end{frame}}}%
%BeginExpansion
\end{frame}%
%EndExpansion

%TCIMACRO{\TeXButton{BeginFrame}{\begin{frame}}}%
%BeginExpansion
\begin{frame}%
%EndExpansion

\QTR{frametitle}{Epimorphisms of posemigroups}

\begin{stepitemize}
\item A posemigroup homomorphism $f:(S,\leq _{S})\longrightarrow (T,\leq
_{T})$ is termed an \textit{epimorphism} if it is right cancellative in the
usual sense of category theory, i.e. for any pair of posemigroup
homomorphisms $g,h:T\longrightarrow W$, $g\circ f=h\circ f\Longrightarrow
g=h $.\bigskip

\item And of course for posemigroups also, $f:(S,\leq _{S})\longrightarrow
(T,\leq _{T})$ is an\smallskip

epimorphism iff $\widehat{dom}_{T}(\func{Im}f)=(T,\leq _{T})$.
\end{stepitemize}

%TCIMACRO{\TeXButton{Transition: Box Out}{\transboxout}}%
%BeginExpansion
\transboxout%
%EndExpansion
%TCIMACRO{\TeXButton{EndFrame}{\end{frame}}}%
%BeginExpansion
\end{frame}%
%EndExpansion

%TCIMACRO{\TeXButton{BeginFrame}{\begin{frame}}}%
%BeginExpansion
\begin{frame}%
%EndExpansion

\QTR{frametitle}{Epimorphisms of posemigroups}

\begin{stepitemize}
\item \textbf{Proposition} Let $\mathcal{V}$ be a variety of absolutely
closed semigroups. Let $\mathcal{V}^{\prime }$ be the variety of
posemigroups obtained by equipping members of $\mathcal{V}$ with all or some
of their compatible orders. Then a posemigroup homomorphism $f$ is epi in $%
\mathcal{V}$ iff it is such in $\mathcal{V}^{\prime }$.
\end{stepitemize}

%TCIMACRO{\TeXButton{Transition: Box Out}{\transboxout}}%
%BeginExpansion
\transboxout%
%EndExpansion
%TCIMACRO{\TeXButton{EndFrame}{\end{frame}}}%
%BeginExpansion
\end{frame}%
%EndExpansion

%TCIMACRO{\TeXButton{BeginFrame}{\begin{frame}}}%
%BeginExpansion
\begin{frame}%
%EndExpansion

\QTR{frametitle}{Epimorphisms of posemigroups}

\textbf{Question} Given that $f:S\longrightarrow T$ is not epi in a class $%
\mathcal{C}$ of semigroups (where $\mathcal{C}$ is different from the
varieties of Higgins' theorem), how (and when) can we find $\leq _{S}$ and $%
\leq _{T}$ so that $f:(S,\leq _{S})\longrightarrow (T,\leq _{T})$ is an
epimorphism of posemigroups?

%TCIMACRO{\TeXButton{Transition: Box Out}{\transboxout}}%
%BeginExpansion
\transboxout%
%EndExpansion
%TCIMACRO{\TeXButton{EndFrame}{\end{frame}}}%
%BeginExpansion
\end{frame}%
%EndExpansion

%TCIMACRO{\TeXButton{BeginFrame}{\begin{frame}}}%
%BeginExpansion
\begin{frame}%
%EndExpansion

\QTR{frametitle}{Acknowledgement and References}

This talk was based on my recent work with Dr. Lauri Tart and on the
following articles.\bigskip

[1] Sohail Nasir: Zigzag theorem for partially ordered monoids. To appear in
Comm. Algebra.\medskip

[2] Sohail Nasir: Absolute closure for pomonoids. Submitted.\bigskip 

I am also thankful to Professor Ross Willard for his valuable
suggestions.\bigskip 
\begin{equation*}
\text{{\huge THANK YOU}}
\end{equation*}

%TCIMACRO{\TeXButton{Transition: Box Out}{\transboxout}}%
%BeginExpansion
\transboxout%
%EndExpansion
%TCIMACRO{\TeXButton{EndFrame}{\end{frame}}}%
%BeginExpansion
\end{frame}%
%EndExpansion

\end{document}
